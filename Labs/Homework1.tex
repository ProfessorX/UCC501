
%%% Local Variables: 
%%% mode: latex
%%% TeX-master: t
%%% End: 
\documentclass[12pt]{article}

% Housekeeping
\usepackage{amsmath}
\usepackage{amssymb}
\usepackage{gensymb}  % For celsius degree symbol

\usepackage{cite}
\usepackage{algorithmic}

% Theorem config
\usepackage{ntheorem}
\newtheorem{Law}{Law}



\usepackage{array}
\usepackage{graphicx}
\graphicspath{{./}{./Figure/}}
\usepackage{url}

% This should be put last.
\usepackage{hyperref}




% And there the document goes.
\begin{document}
\title{Sustainable Energy Homework 1 Solutions}
\author{Yanan Xiao}
\maketitle{}
% Fuck that.

\section{Question 1}
\label{sec:question-1}

Solution: Since we do not consider any heat transfer and neglect
kinetic energy except out of nozzle, we have the following:

\begin{itemize}
\item Compressor: $Q~=~\Delta h + W_{t}~=~ 0$, therefore we have $W_{t}~ =~
  -\Delta h$.\\
  Evaluate the equation above we have $W_{t}~=~-539.96~(KJ\cdot
  Kg^{-1})$
\item Turbine: For the same reason, we have $Q~=~\Delta h +
  W_{t}~=~ 0$. Therefore $W_{t}~=~ -(933.15 - 1635.8)~=~
  702.65~(KJ\cdot Kg^{-1})$
\item Nozzle: As analyzed above, we can reach this equation
  $Q~=~\Delta h + E_{k}~=~ 0$ where $E_{k}$ stands for the kinetic
  energy of gases. Thus, $\frac{1}{2}v^{2}~=~ -\Delta h$, evaluate
  this equation we have exit velocity\\ $v~=~\sqrt{2*283.62*10^{3}}~=~
  753.2~m/s$
\item Compressor: If the turbine specific work is completely used to
  drive the compressor, we have $Q~=~\Delta h + W_{t}$ where
  $W_{t}=0$. Thus, $Q~=~\Delta h~=~ 1635.8 -
  800.28~=~835.52~(KJ/Kg)$.\\
  So the jet efficiency is $\eta~=~\frac{0.5*v^{2}}{q}$, evaluate this
  equation we have $\eta~=~33.9\%$.
\item Enthalpy: When $W_{turbine}~=~W_{compressor}$, we have
  $1635.8-W_{tout}=539.96$, so the turbine outlet enthalpy is
  $1095.84~(KJ/Kg)$ 
\end{itemize}

\section{Question 2}
\label{sec:question-2}
% Life is awesome in the United Arab Emirates!
% $E_{in}-E_{out}~=~\Delta E_{system}\quad (kJ)$

From this page \url{https://en.wikipedia.org/wiki/R-410A} we know the
physical properties of R-410A. Since the boiling point for R-410A is
$-48.5\celsius$, so it will stay in gas form all the circulation.
\par
For easy calculation we assume the volume of R-410A is constant. Since
the flow rate is $0.05~Kg/s$, we use flow time $t~=~1s$.
\begin{itemize}
\item Thus $m~=~0.05 * 1~=~0.05~Kg$
  \begin{equation}
    \label{eq:1}
    W_{comp}-Q_{comp}~=~m*(h_{state6}-h_{state1})
  \end{equation}
  This is inferred from 1st law of
  thermaldynamics. Evaluate~\eqref{eq:1} we have heat transfer from the
  compressor is $(377-384)*0.05~=~4.65~KJ$.
\item Following a similar scheme, we have the heat transfer in the
  condenser is $(134-367)*0.05~=~11.65~KJ$
\item As above, the expansion valve does not change the enthalpy
  during a throttling process, so we have
  \begin{equation}
    \label{eq:2}
    Q_{evap}~=~m*(h_{state5}-h_{state4})
  \end{equation}
  Evaluate~\eqref{eq:2} we have $Q_{evap}~=~7.3~KJ$
\item From the question we have $W_{in}~=~P_{comp}*t~=~5~KJ$, and
  $Q_{cond}=11.65~KJ$. Thus we have,
  \begin{equation}
    \label{eq:3}
    COP_{HP}~=~\frac{Q_{cond}}{W_{in}=W_{comp}}
  \end{equation}
  Evaluate the above equation we have $COP~=~2.33$
\end{itemize}

\section{Question 3}
\label{sec:question-3}

From general physics background we have:
\begin{itemize}
\item $m_{water}~=~100*10^{-3}*0.001043^{-1}~=~95.87~Kg$
\item For the turbine we can infer $Q~=~\Delta h+ W_{t}$, then
  \begin{equation}
    \label{eq:4}
    W_{t}~=~-\Delta h~=~-(2675.5-2763.5)~=~88 KJ/Kg
  \end{equation}
  Therefore the total turbine work is $88 * 95.87~=~9436.56~KJ$
\item For burner we have $Q~=~\Delta h$.
  Evaluate~\eqref{eq:4} we have total heat equals
  $(2763.5-417.36)*95.87~=~224924.4~KJ$ 
\end{itemize}

\section{Question 4}
\label{sec:question-4}

\begin{Law}
  It is impossible to construct a device that operates in a cycle and
  produces no effect other than the transfer of heat from a
  lower-temperature body to a higher-temperature body. 
\end{Law}
From this expression of $2^{nd}$ law of thermaldynamics we can easily
reach the conclusion that this machine is \textbf{possible but
  irreversible}.

\section{Question 5}
\label{sec:question-5}

For the three solids mentioned in the question, we have:
\begin{itemize}
\item Silicon: $Q~=~C*m*\Delta T$, where $C$ is the heat capacity (the
  rest of this question use same notation), $m$ is the mass of solid,
  and $\Delta T$ stands for the temperature change.\\
  Evaluate this equation we have
  $Q_{silicon}~=~50*10^{-3}*55*0.712*10^{3}~=~1958~J$
\item Copper: Following the same algorithm, we have
  $Q_{copper}~=~20*0.385*55~=~423.5~J$
\item Polyvinyl chloride: $Q_{poly}~=~1.0049*50*55~=~2763.475~J$
\end{itemize}

\end{document}
