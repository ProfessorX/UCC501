
%%% Local Variables: 
%%% mode: latex
%%% TeX-master: t
%%% End: 

\documentclass[12pt]{article}

% Housekeeping
\usepackage{amsmath}
\usepackage{amssymb}
\usepackage{gensymb}  % For celsius degree symbol

\usepackage{cite}
\usepackage{algorithmic}

% Theorem config
\usepackage{ntheorem}
\newtheorem{Law}{Law}

\usepackage{array}
\usepackage{listings}
\usepackage{graphicx}
\graphicspath{{./}{./Figure/}}
\usepackage{url}

% This should be put last.
\usepackage{hyperref}



\begin{document}
% And there the document goes.

\title{UCC501 Homework 2 Solutions}
\author{Yanan Xiao\\\url{yxiao@masdar.ac.ae}}
\maketitle{}
% Fuck that.

\section{Energy Scales, CO2  Emissions and Renewables}
\label{sec:question-1}

\subsection{Energy Scales}
\label{sec:energy-scales}

% Life is fucking awesome in United Arab Emirates!
\begin{itemize}
\item Natural gas consumption year 2012: in Billion Cubic Feet:
  2235.169. 
\item For natural gases btu we have
  \begin{equation}
    \label{eq:1}
    1cf\rightarrow 1027 Btu
  \end{equation}
  source:
  \url{http://www.aga.org/KC/ABOUTNATURALGAS/ADDITIONAL/Pages/HowtoMeasureNaturalGas.aspx},
  use MATLAB to convert units, we have
  \begin{equation}
    \label{eq:2}
    2012~Yearly~btu~=~2235.169*10^{9}*1027
  \end{equation}
  Evaluate the equation we get $2.2955*10^{15}~Btu$, divided by
  millions we get $2.2955*10^{9}~MBtu$.
\item From this unit conversion site
  \url{http://www.eia.gov/cfapps/ipdbproject/docs/units.cfm} we have
  \begin{equation}
    \label{eq:3}
    MTOE~=~MBtu * 0.02520 * 10^{-6}
  \end{equation}
  Evaluate the above equation we have $57.8471 MTOE$
\item Similarly we have $1 MBtu~=~1.05506*10^{9} Joules$ and $1
  KWh~=~3.6*10^{6} Joules$, thus
  \begin{equation}
    \label{eq:4}
    2012~Yearly~GKWh~=~2.2955*10^{9} * 1.05506 *10^{9} / (3.6*10^{6} *
    10^{9}) 
  \end{equation}
  Evaluate the above equation we have $672.7527~GKWh$

\end{itemize}

\subsection{CO2 Emissions}
\label{sec:co2-emissions}

\begin{itemize}
\item Assume complete combustion, we have $2235.169~Bcf$ natural gas
  burned in UAE for year 2012. Assume under normal temperature and
  pressure the natural gas is measured, we google the bold texts
  \textbf{natural gas density} and get the following:
\begin{verbatim}
Density
0.656 g/L at 25 °C, 1 atm; 0.716 g/L at 0 °C, 1 atm;
0.42262 g cm−3; (at 111 K)
\end{verbatim}
  Since we assume normal density, we use $\rho~=~0.656 g/L~=~656
  kg/m^{3}$. Thus we have the total mass of CO2 burned in year 2012
  \begin{equation}
    \label{eq:5}
    Mass~of~CH_{4}~=~4.1525*10^{13} kg
  \end{equation}
  Since the background is complete combustion, from chemistry we have
  \begin{equation}
    \label{eq:6}
    CH_{4}~+~2O_{2}~=~ CO_{2}~+~2H_{2}O
  \end{equation}
  and introduce a concept from chemistry, \textbf{molecular weight} we
  have 
  \begin{equation}
    \label{eq:7}
    Mass~of~CO_{2}~=~m_{CH_{4}}*\frac{M_{CO_{2}}}{M_{CH_{4}}}
  \end{equation}
  where $m_{CH_{4}}$ is the mass, the $M_{CH_{4}}$ is the molecular
  weight. This equation is deducted by \textbf{carbon equilibrium}
  under complete combustion.\\
  Evaluate equation (\ref{eq:7}) we have $m_{CO_{2}}~=~1.1419 *
  10^{14} kg$

\end{itemize}

\subsection{Renewables}
\label{sec:renewables-1}

Since we mentioned \textbf{in total} in the question, my answer would
follow the \emph{total} CO2 generated by burning natural gas.
\par
Thus $m1_{co_{2}}~=~0.05 * m_{CO_{2}}$, evaluating it we have
$m1_{co_{2}}~=~5.7097 * 10^{12} kg$.
\begin{itemize}
\item Set the size of such solar plant is $x~MW$, then from
  \textbf{Energy Conservation Law} we have
  \begin{equation}
    \label{eq:8}
    Energy_{solar}~=~Energy_{gas~useful} * 0.05
  \end{equation}
\item For energy generated by a solar plant's lifetime, we have 
  \begin{equation}
    \label{eq:9}
    E_{solar}~=~x * 3.6 * 10^{9} * 1700 * 25
  \end{equation}
\item For energy generated by burning natural gas, we have the
  secondary energy as follows:
  \begin{equation}
    \label{eq:10}
    E_{gas_useful}~=~E_{gas} * 0.40 * 0.92
  \end{equation}
  The $E_{gas}$ can be calculated and/or converted from the
  above. Because we get 2012 yearly gas btu $2.2955 * 10^{15}$,
  convert that to Joule unit
  \begin{equation}
    \label{eq:11}
    E_{gas}~=~2.2955 * 10^{9} * 1.05506 * 10^{9}
  \end{equation}
  Put all these values into (\ref{eq:8}) we have $x~=~291.2624$. Thus
  the size of such a plant is $291.2624\simeq 300~MW$. (I put 300MW
  here for \emph{industrial term}.)
\end{itemize}


\section{Economic Analysis}
\label{sec:economic-analysis}

% Life is fucking awesome in the United Arab Emirates. Let the killing
% begin. 

\subsection{LCOE Calculation}
\label{sec:lcoe-calculation}

From section \ref{sec:question-1} we have the scale of solar farm is
close to 300MW. So to be practical we here use \texttt{Scale =
  300MW}.
\begin{itemize}
\item The whole calculation is done through MATLAB programming,
  therefore, to save pages, we do not repeat the values. We list
  relevant equations here.
% \item Life is fucking awesome in the United Arab Emirates!
\item
  \begin{equation}
    \label{eq:12}
    Total~Capital~Cost~=~Unit~Capital~Cost * Solar~Farm~Scale
  \end{equation}
\item From the values we have, we can use WACC as a \textbf{pretty
    rough} estimation of \textbf{project annual discount rate}. It is
  also listed in the UAE bureau of statistics the expected CPI of next
  5 years, we would like to do another estimate adding this value, if
  possible. 
  \begin{equation}
    \label{eq:13}
    WACC~=~(debt~return*debt~ratio)+(equity~return*equity~ratio)
  \end{equation}
\item The remaining part of question is self-evident in the codes as
  can be seen in the reference link. Following the program, we have
  \begin{equation}
    \label{eq:14}
    Levelized~cost~=~0.0412~\$/KWh
  \end{equation}

\end{itemize}

\subsection{IRR Calculation}
\label{sec:irr-calculation}

There is something wrong either with the question or with my
calculation. But I followed the slides so...Well, this is Masdar.


\section{Vehicle Kinetics}
\label{sec:vehicle-kinetics}

\subsection{Propulsion Power}
\label{sec:propulsion-power}

\begin{itemize}
\item From the question we have all the values needed, so just
  evaluate the equation we have $P_{v}~=~6.2321*10^{3}~W$. The whole
  calculation code is shown in reference as well as the link.
\end{itemize}

\subsection{Annual Vehicle Energy Consumption}
\label{sec:annu-vehicle-energy}
% 8.4 Litre/ 100km
% H/C = 2.25
\begin{itemize}
\item For this question I tried to replace the $d$ (distance between
  stops) with $40,000~Km$ but it did not work out as
  expected. Therefore we need to consider this problem in a
  \emph{macro} scale.
\item Since the given mass is $1440Kg$ we can infer that the type of
  engine is a $2.5L$ one. From the manual and
  Wikipedia\footnote{\url{http://en.wikipedia.org/wiki/Toyota_Camry_(XV50)}}
  we have the combined fuel consumption of around
  $8L/100Km$. Moreover, we can get the standard reference density of
  marketable gasoline $0.755Kg/L$. 
\item The rest part is and only is calculation, and that is reflected
  in the source code.
\end{itemize}

\subsection{Abu Dhabi Car Annual}
\label{sec:abu-dhabi-car}

\begin{itemize}
% \item Life is fucking awesome in UAE.
\item The data used in this question can be referred from heat of
  combustion
  Wikipedia\footnote{\url{http://en.wikipedia.org/wiki/Heat_of_combustion}}. We
  use $47.30MJ/Kg$ for gasoline heating value.
\item The process of calculation is a bit tedious but self-evident in
  codes, we can get $1.9046*10^{7}MWh$ energy requirement for an
  estimated 600,000 registered cars in Abu Dhabi. The figure is also
  calculated under full  conversion efficiency.
\item For $CO_{2}$ emissions, we get $4.5236*10^{9}Kg$.
\end{itemize}

\subsection{Panel Installment}
\label{sec:panel-installment}

\begin{itemize}
\item According to all the assumptions and calculations above, we get
  the annual installment of solar power, in $MWh$ is $1.1204*10^{3}MWh$
\end{itemize}

\section{Reference}
\label{sec:reference}


\end{document}

TODO: 
#1 Add reference link to your codes since the embedding is ugly.

