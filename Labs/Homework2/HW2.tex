
%%% Local Variables: 
%%% mode: latex
%%% TeX-master: t
%%% End: 

\documentclass[12pt]{article}

% Housekeeping
\usepackage{amsmath}
\usepackage{amssymb}
\usepackage{gensymb}  % For celsius degree symbol

\usepackage{cite}
\usepackage{algorithmic}

% Theorem config
\usepackage{ntheorem}
\newtheorem{Law}{Law}

\usepackage{array}
\usepackage{listings}
\usepackage{graphicx}
\graphicspath{{./}{./Figure/}}
\usepackage{url}

% This should be put last.
\usepackage{hyperref}



\begin{document}
% And there the document goes.

\title{UCC501 Homework 2 Solutions}
\author{Yanan Xiao\\\url{yxiao@masdar.ac.ae}}
\maketitle{}
% Fuck that.

\section{Energy Scales, CO2  Emissions and Renewables}
\label{sec:question-1}

\subsection{Energy Scales}
\label{sec:energy-scales}

% Life is fucking awesome in United Arab Emirates!
\begin{itemize}
\item Natural gas consumption year 2012: in Billion Cubic Feet:
  2235.169. 
\item For natural gases btu we have
  \begin{equation}
    \label{eq:1}
    1cf\rightarrow 1027 Btu
  \end{equation}
  source:
  \url{http://www.aga.org/KC/ABOUTNATURALGAS/ADDITIONAL/Pages/HowtoMeasureNaturalGas.aspx},
  use MATLAB to convert units, we have
  \begin{equation}
    \label{eq:2}
    2012~Yearly~btu~=~2235.169*10^{9}*1027
  \end{equation}
  Evaluate the equation we get $2.2955*10^{15}~Btu$, divided by
  millions we get $2.2955*10^{9}~MBtu$.
\item From this unit conversion site
  \url{http://www.eia.gov/cfapps/ipdbproject/docs/units.cfm} we have
  \begin{equation}
    \label{eq:3}
    MTOE~=~MBtu * 0.02520 * 10^{-6}
  \end{equation}
  Evaluate the above equation we have $57.8471 MTOE$
\item Similarly we have $1 MBtu~=~1.05506*10^{9} Joules$ and $1
  KWh~=~3.6*10^{6} Joules$, thus
  \begin{equation}
    \label{eq:4}
    2012~Yearly~GKWh~=~2.2955*10^{9} * 1.05506 *10^{9} / (3.6*10^{6} *
    10^{9}) 
  \end{equation}
  Evaluate the above equation we have $672.7527~GKWh$

\end{itemize}

\subsection{CO2 Emissions}
\label{sec:co2-emissions}

\begin{itemize}
\item Assume complete combustion, we have $2235.169~Bcf$ natural gas
  burned in UAE for year 2012. Assume under normal temperature and
  pressure the natural gas is measured, we google the bold texts
  \textbf{natural gas density} and get the following:
\begin{verbatim}
Density
0.656 g/L at 25 °C, 1 atm; 0.716 g/L at 0 °C, 1 atm;
0.42262 g cm−3; (at 111 K)
\end{verbatim}
  Since we assume normal density, we use $\rho~=~0.656 g/L~=~656
  kg/m^{3}$. Thus we have the total mass of CO2 burned in year 2012
  \begin{equation}
    \label{eq:5}
    Mass~of~CH_{4}~=~4.1525*10^{13} kg
  \end{equation}
  Since the background is complete combustion, from chemistry we have
  \begin{equation}
    \label{eq:6}
    CH_{4}~+~2O_{2}~=~ CO_{2}~+~2H_{2}O
  \end{equation}
  and introduce a concept from chemistry, \textbf{molecular weight} we
  have 
  \begin{equation}
    \label{eq:7}
    Mass~of~CO_{2}~=~m_{CH_{4}}*\frac{M_{CO_{2}}}{M_{CH_{4}}}
  \end{equation}
  where $m_{CH_{4}}$ is the mass, the $M_{CH_{4}}$ is the molecular
  weight. This equation is deducted by \textbf{carbon equilibrium}
  under complete combustion.\\
  Evaluate equation (\ref{eq:7}) we have $m_{CO_{2}}~=~1.1419 *
  10^{14} kg$

\end{itemize}

\subsection{Renewables}
\label{sec:renewables-1}

Since we mentioned \textbf{in total} in the question,  my answer 
\end{document}

