%%% Local Variables:
%%% mode: latex
%%% TeX-master: t
%%% End:
\documentclass[twocolumn]{article}


% Packages
% \usepackage{fancyhdr}
\usepackage{amsmath}
\usepackage{amssymb}
\usepackage[margin=10mm]{geometry}
\usepackage{graphicx}



% Housekeeping
\pagestyle{empty}


\begin{document}
\begin{itemize}
% \item Life is fucking awesome in the United Arab Emirates!
\item For policy makers, they care: sustainability, and the effect of
  policies on: economy, environment and equity.
\item \textbf{Discount Rate} Generally, you can work with constant \$ if you deal with
  project costs. Implies minimum acceptable profitability.
\item \textbf{Raising Capital} Debt (borrowing) Equity (selling parts
  of the company)
\item \textbf{WACC} Can be used as a metric, a reasonable
  approximation of DR; But it does neither account for project risk
  nor opportunity cost.
\item \textbf{Internal Rate of Return} Discount rate for which NPV is
  $0$. In the context of savings and loans the IRR is also called the
  effective interest rate. Because the internal rate of return is a
  rate quantity, it is an indicator of the efficiency, quality, or
  yield of an investment. This is in contrast with the net present
  value, which is an indicator of the value or magnitude of an
  investment. 
\item \textbf{LCOE} Levelized Energy Cost is the price at which
  electricity must be generated from a specific source to \textbf{break even}
  over the lifetime of the project. It is an economic assessment of
  the cost of the energy-generating system including all the costs
  over its lifetime: initial investment, operations and maintenance,
  cost of fuel, cost of capital, and is very useful in calculating the
  costs of generation from different sources.
\item Issues with LCOE: However, care should be taken in comparing
  different LCOE studies and the sources of the information as the
  LCOE for a given energy source is \textbf{highly dependent} on the
  assumptions, financing terms and technological deployment analyzed.
\item \textbf{Break Even} is the point at which total cost and total
  revenue are equal.
\item \textbf{Net Present Value} In finance, the net present value
  (NPV) of a time series of cash flows, both incoming and outgoing, is
  defined as the sum of the present values (PVs) of the individual
  cash flows of the same entity. NPV can be described as the
  “difference amount” between the sums of discounted: cash inflows and
  cash outflows. It compares the present value of money today to the
  present value of money in the future, taking inflation and returns
  into account. 
\item Yet some more notes on NPV.
  \begin{itemize}
  \item The investment horizon of all possible investment projects
    considered are equally acceptable to the investor.
  \item The 10\% discount rate is the appropriate (and stable) rate to
    discount the expected cash flows from each project being
    considered. Each project is assumed equally speculative. 
  \item the shareholders can't get above a 10\% return on their money
    if they were to directly assume an equivalent level of risk.  
  \end{itemize}
\end{itemize}
\end{document}
